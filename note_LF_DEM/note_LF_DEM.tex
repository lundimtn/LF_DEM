%#BIBTEX semiautotex.sh -b note_LF_DEM
\documentclass[12pt]{article}
\usepackage{amsmath,amssymb}
\usepackage{bm}
%\usepackage{hyperref}
\usepackage[pdftex]{graphicx}	% required for `\includegraphics' (yatex added)
\usepackage{datetime}
\usepackage{natbib}
%\usepackage{bm}% bold math
\usepackage[pdftex,bookmarks,colorlinks]{hyperref}

\oddsidemargin -1.2cm
\evensidemargin -1.2cm
\textwidth 18cm
\headheight 1.0in
\topmargin -3.5cm
\textheight 22cm
\bibliographystyle{unsrtnat}
\newcommand{\figref}[1]{\figurename ~\ref{#1}}
\newcommand{\tens}[1]{\bm{\mathsf{#1}}}
\title{Simulation model for concentrated suspension\\
\emph{lubrication} and \emph{contact force}}
\date{\shortdate\today \, \ampmtime }
\author{R. Seto, R. Mari}
\begin{document}
\maketitle

\section{Motivation}
\subsection*{Ball and Merlose model}

\subsubsection*{Ball-1995}

\citet{Ball_1995} ``Lubrication breakdown in hydrodynamic simulations of concentrated colloids''
reported the difficulty of the simulation including lubrication force.
Their model is FT-level ($6N$-vectors).
\begin{equation}
 \bm{F}_{\mathrm{H}} = - \tens{R} \bm{V}
\end{equation}
is used for the over damped motion.
\begin{equation}
 \bm{F}_{\mathrm{H}} = \bm{F}_{\mathrm{C}} = 0.
\end{equation}
%
The leading order of the pair-wise squeeze hydrodynamic force
was used.
\begin{equation}
 \bm{f}_i = 
- \sum_j \frac{3 \pi \mu}{8 h_{ij}} 
\left\{
(\bm{v}_i - \bm{v}_j)\cdot \bm{n}_{ij}
\right\} \bm{n}_{ij}\label{125344_18Dec12}
\end{equation}
%
They reported the singular results obtained by this model.
%
The minimum gap and viscosity 
depend on simulation detal, such as time step.
%
They concluded
``The ideal problem of smooth hard spheres
in a Newtonian fluid under simple shear appears
to be singular in nature''.
%

%%%%%%%%%%%%%%%%%%%%%%%%%%%%%%%%%%%%%%%%%%%%%%%%%%%%%%%%%%%%%%%%%%%%%%%%%%%%%%%%%%%%%%%%%%%%%%%%%%%%

We are dubious about their modeling.
%
Their lubrication model \eqref{125344_18Dec12}
is not the same as the leading order of \citet{Jeffrey_1992}.

%%%%%%%%%%%%%%%%%%%%%%%%%%%%%%%%%%%%%%%%%%%%%%%%%%%%%%%%%%%%%%%%%%%%%%%%%%%%%%%%%%%%%%%%%%%%%%%%%%%%

In \citet{Ball_1995}, 
the torque is not written explicitly.
%

\subsubsection*{Melrose-1995}

The article by \citet{Melrose_1995}
titled ``\emph{The Pathological Behaviour of Sheared Hard Spheres with Hydrodynamic Interactions}'' also discussed the difficulty with the same level.

\begin{quote}
\emph{The divergent squeeze terms in principle prevent overlap of particles,
 but computations with fured time steps lead to overlap. }
\end{quote}

\subsubsection*{Melrose-1996}

In \citet{Melrose_1996}
titled ``\emph{Continuous shear thickening and colloid surfaces}'',
they moved on the problem of rheology 
by using the model.  



\subsubsection*{Ball-1997}

Simulation method was reviewed  
by \citet{Ball_1997},
titled 
``\emph{A simulation technique for many spheres 
in quasi-static motion under frame-invariant pair drag and Brownian forces}''.

The equation (12) in the paper is 
\begin{equation}
 - \tens{R} (\bm{V}-\bm{V}_0)
 + \bm{F}^{\mathrm{C}}
 + \bm{F}^{\mathrm{B}}
 - \tens{R} \bm{V}_0 = 0
\end{equation}

\begin{quotation}
\emph{
 The final term in Eq.~(12) is often termed the shear tensor. 
Here it is evaluated at the frame-invariant pair level of the basic approximation. 
(If the shear tensor and the matrix R are formulated separately,
 as, for example, in moment expansions, 
it may be that frame invariance is broken and this should be checked for in practice.)
}
\end{quotation}

\subsubsection*{Farr-1997}

\citet{Farr_1997} 
``\emph{Kinetic theory of jamming in hard-sphere startup flows}''
discussed the physics.

\subsubsection*{Silbert-1997}
\citet{Silbert_1997}
``Colloidal microdynamics: Pair-drag simulations of model-concentrated aggregated systems''


\subsubsection*{Silbert-1999-1}
\citet{Silbert_1999}
``\emph{A structural analysis of concentrated, aggregated colloids under flow}''
added an attractive force to cause 
aggregation and analyzed the structure.

\subsubsection*{Silbert-1999-2}
\citet{Silbert_1999a}
``Stress distributions in flowing aggregated colloidal suspensions''
also include the attractive force.
%
They studied force correlations.

\subsubsection*{Catherall-2000}

\citet{Catherall_2000}
``Shear thickening and order-disorder effects in concentrated colloids at high shear rates''
discussed the order formation and the stability.

\subsubsection*{Melrose-2004-1}

\citet{Melrose_2004a}
\emph{Continuous shear thickening transitions in model concentrated colloids
--The role of interparticle forces}

They still used \eqref{125344_18Dec12}.

\subsubsection*{Melrose-2004-2}

\citet{Melrose_2004}
\emph{``Contact networks'' in continuously shear thickening colloids}

\subsection*{Stokesian dynamics}

\begin{itemize}
  \item 
  \citet{Bossis_1984}:
  Dynamic simulation of sheared suspensions. I. General method
  \item
  \citet{Brady_1985}:
  The rheology of concentrated suspensions of spheres in simple shear flow by numerical simulation
  \item 
  \citet{Brady_1988}:
  Stokesian dynamics
  \item
  \citet{Bossis_1988}:
  Shear-induced structure in colloidal suspensions I. numerical simulation

  \item \citet{Phung_1996}:
  Stokesian dynamics simulation of Brownian suspensions
  \item \citet{Brady_1997}: 
  Microstructure of strongly sheared suspensions and its impact on rheology and diffusion

  \item \citet{Foss_2000}:
  Structure, diffusion and rheology of {B}rownian suspensions by {S}tokesian dynamics simulation
  \item \citet{Sierou_2002}:
  Rheology and microstructure in concentrated noncolloidal suspensions
  \item \citet{Morris_2002}:
  Microstructure from simulated Brownian suspension flows at large shear rate
  \item \citet{Sierou_2004}:
  Shearinduced selfdiffusion in noncolloidal suspensions
  \item \citet{Wagner_2009}:
  Shear thickening in colloidal dispersions
  \item \citet{Morris_2009}:
  A review of microstructure in concentrated suspensions 
  and its implications for rheology and bulk flow
 \item 
\end{itemize}

\subsection*{Experiments}

\begin{itemize}
 \item \citet{Bender_1996}:
Reversible shear thickening in monodisperse and bidisperse colloidal dispersions
 \item \citet{Melrose_1996}:
Continuous shear thickening and colloid surfaces
 \item \citet{Maranzano_2001}:
The effects of particle size on reversible shear thickening of concentrated colloidal dispersions
 \item \citet{Maranzano_2001a}:
The effects of interparticle interactions and particle size 
on reversible shear thickening: Hard-sphere colloidal dispersions
 \item \citet{Maranzano_2002}
Flow-small angle neutron scattering measurements 
of colloidal dispersion microstructure evolution through the shear thickening 
 \item \cite{Lee_2003}
Dynamic properties of shear thickening colloidal suspensions

 \end{itemize}


\section{Leading terms of two-body exact solution}

\subsection*{The leading lubrication forces}


Interaction between particles $\alpha$ and $\beta$
can be written in a linear form:
\begin{equation}
 \begin{pmatrix}
  \bm{F} \\ \bm{T} \\ \bm{S}
 \end{pmatrix}
 =
- \mu
\begin{pmatrix}
 \tens{A} &  \tilde{\tens{B}} &  \tilde{\tens{G}} \\
 \tens{B} &  \tens{C} &  \tilde{\tens{H}} \\
 \tens{G} &  \tens{H} &  \tens{M} \\
\end{pmatrix}
 \begin{pmatrix}
  \bm{U}-\bm{U}_{\infty} \\ \bm{\Omega} - \bm{\Omega}_{\infty}
 \\ -\bm{E}_{\infty}
 \end{pmatrix}
\end{equation}
The vectors consist of two elements.
\begin{equation}
 \bm{F}
= 
\begin{pmatrix}
 \bm{F}_\alpha \\  \bm{F}_\beta
\end{pmatrix}
,\quad
 \bm{T}
= 
\begin{pmatrix}
 \bm{T}_\alpha \\  \bm{T}_\beta
\end{pmatrix}
,\quad
 \bm{S}
= 
\begin{pmatrix}
 \bm{S}_\alpha \\  \bm{S}_\beta
\end{pmatrix}
,\quad
 \bm{U}
= 
\begin{pmatrix}
 \bm{U}_\alpha \\  \bm{U}_\beta
\end{pmatrix}
,\quad
 \bm{\Omega}
= 
\begin{pmatrix}
 \bm{\Omega}_\alpha \\  \bm{\Omega}_\beta
\end{pmatrix}
\end{equation}
The explicit form of the matrix $\tens{A}$ 
is as follows
\begin{equation}
 \tens{A}
 =
\begin{pmatrix}
  \tens{A}_{\alpha\alpha} & \tens{A}_{\alpha\beta} \\
  \tens{A}_{\beta\alpha} & \tens{A}_{\beta\beta} 
\end{pmatrix}.
\end{equation}
%
If we consider only the leading terms for the nearly contacting particles,
only $\tens{A}$ and $\tens{G}$ give the leading contribution to the forces.
\begin{equation}
 \begin{pmatrix}
  \bm{F}_{\alpha} \\
  \bm{F}_{\beta} 
 \end{pmatrix}
\approx
- \mu
\begin{pmatrix}
\tens{A}_{\alpha\alpha} &
\tens{A}_{\alpha\beta}  \\
\tens{A}_{\beta\alpha}  &
\tens{A}_{\beta\beta}  
\end{pmatrix}
 \begin{pmatrix}
  \bm{U}_{\alpha} -  \bm{U}_{\alpha}^{\infty}\\
  \bm{U}_{\beta} -  \bm{U}_{\beta}^{\infty}
 \end{pmatrix}
 + \mu
\begin{pmatrix}
\tilde{\tens{G}}_{\alpha\alpha} &
\tilde{\tens{G}}_{\alpha\beta}  \\
\tilde{\tens{G}}_{\beta\alpha}  &
\tilde{\tens{G}}_{\beta\beta}  
\end{pmatrix}
 \begin{pmatrix}
\bm{E}_{\infty} \\ \bm{E}_{\infty}
\end{pmatrix}
\end{equation}


\begin{align*}
F^{\alpha}_i
&= 
- \mu A^{\alpha\alpha}_{ij}
(\bm{U}_{\alpha}-\bm{U}_{\alpha}^{\infty})_j
- \mu A^{\alpha\beta}_{ij}
(\bm{U}_{\beta}-\bm{U}_{\beta}^{\infty})_j
+ \mu \tilde{G}^{\alpha\alpha}_{ijk} \bm{E}^{\infty}_{jk}
+ \mu \tilde{G}^{\alpha\beta}_{ijk} \bm{E}^{\infty}_{jk} \\
&=
- 6 \pi\mu a_{\alpha} \hat{A}^{\alpha\alpha}_{ij}
(\bm{U}_{\alpha}-\bm{U}_{\alpha}^{\infty})_j
- 3 \pi\mu (a_{\alpha}+a_{\beta})\hat{A}^{\alpha\beta}_{ij}
(\bm{U}_{\beta}-\bm{U}_{\beta}^{\infty})_j \\
&\quad
+ 6 \pi\mu a_{\alpha} \hat{\tilde{G}}^{\alpha\alpha}_{ijk} \bm{E}^{\infty}_{jk}
+ 3 \pi\mu (a_{\alpha}+a_{\beta}) \hat{\tilde{G}}^{\alpha\beta}_{ijk} \bm{E}^{\infty}_{jk} \\
%%%%%%%%%%%%%%%%%%%%%%%%%%%%%%%%%%%%%%%%%%%%%%%%%
&=
- 6 \pi\mu a_{\alpha} X_{\alpha\alpha}^{A} n_i n_j
(\bm{U}_{\alpha}-\bm{U}_{\alpha}^{\infty})_j
- 3 \pi\mu (a_{\alpha}+a_{\beta}) X_{\alpha\beta}^{A} n_i n_j
(\bm{U}_{\beta}-\bm{U}_{\beta}^{\infty})_j \\
&\quad
+ 6 \pi\mu a_{\alpha} X_{\alpha\alpha}^{G} 
\left(n_i n_j - \frac{\delta_{ij}}{3}  \right)n_k \bm{E}^{\infty}_{jk}
+ 3 \pi\mu (a_{\alpha}+a_{\beta}) 
X_{\alpha\beta}^{G} \left(n_i n_j - \frac{\delta_{ij}}{3} \right)n_k \bm{E}^{\infty}_{jk} \\
%%%%%%%%%%%%%%%%%%%%%%%%%%%%%%%%%%%%%%%%%%%
&=
- 6 \pi\mu a_{\alpha} \frac{g_1(\lambda)}{\xi} n_i n_j
(\bm{U}_{\alpha}-\bm{U}_{\alpha}^{\infty})_j
+ \frac{6}{1+\lambda} \pi\mu (a_{\alpha}+a_{\beta})\frac{g_1(\lambda)}{\xi} n_i n_j
(\bm{U}_{\beta}-\bm{U}_{\beta}^{\infty})_j 
\\
&\quad
+ 9 \pi \mu a_{\alpha} \frac{g_1(\lambda)}{\xi} 
\left(n_i n_j - \frac{\delta_{ij}}{3} \right)n_k \bm{E}^{\infty}_{jk}
-  
\frac{18 \pi}{(1+\lambda)^2} 
\mu (a_{\alpha}+ a_{\beta})
\frac{g_1(\lambda)}{\xi} 
\left(n_i n_j - \frac{\delta_{ij}}{3} \right) n_k \bm{E}^{\infty}_{jk} 
\end{align*}

monodisperse $a_\alpha = a_{\beta}=a$, $g_1(1)=1/4$.
\begin{align*}
 F^{\alpha}_i
&= 
- 6 \pi\mu a \frac{1}{4\xi} n_i n_j
(\bm{U}_{\alpha}-\bm{U}_{\alpha}^{\infty})_j
+ 6 \pi\mu a\frac{1}{4\xi} n_i n_j
(\bm{U}_{\beta}-\bm{U}_{\beta}^{\infty})_j 
\\
&\quad
+ \frac{9}{4} \pi \mu a \frac{1}{\xi} 
\left(n_i n_j - \frac{\delta_{ij}}{3} \right) n_k \bm{E}^{\infty}_{jk}
-  
\frac{9}{4} \pi \mu a
\frac{1}{\xi} 
\left(n_i n_j - \frac{\delta_{ij}}{3} \right)n_k \bm{E}^{\infty}_{jk} \\
&= 
- \frac{3}{2} \pi\mu a \frac{1}{\xi} n_i n_j
(\bm{U}_{\alpha}-\bm{U}_{\alpha}^{\infty})_j
+ \frac{3}{2} \pi\mu a\frac{1}{\xi} n_i n_j
(\bm{U}_{\beta}-\bm{U}_{\beta}^{\infty})_j 
+ 0 \Leftarrow ???
\end{align*}


%\begin{align}
%\frac{ F^{\alpha}_i}{F_0}
%&= 
%-  \frac{1}{4\xi} n_i n_j
%(\bm{U}_{\alpha}-\bm{U}_{\alpha}^{\infty})_j
%+ \frac{1}{4\xi} n_i n_j
%(\bm{U}_{\beta}-\bm{U}_{\beta}^{\infty})_j 
%\\
%&\quad
%+ \frac{3}{8\xi}
%(n_i n_j - 1/3 \delta_{ij}) \bm{E}^{\infty}_{jk}
%-  
%\frac{3}{8\xi}
%(n_i n_j - 1/3 \delta_{ij}) \bm{E}^{\infty}_{jk} 
%\end{align}


\paragraph{Symmetries:}
2-indeces:
\begin{equation}
 A_{ij}^{\alpha\beta} =  A_{ji}^{\beta\alpha}
\end{equation}
3-indeces:
\begin{equation}
 G_{ijk}^{\alpha\beta} =   G_{jik}^{\alpha\beta} , \quad 
 G_{iik}^{\alpha\beta} =   H_{iik}^{\alpha\beta}  = 0, \quad 
 \tilde{G}_{ijk}^{\alpha\beta}  =  G_{jki}^{\beta\alpha} 
\end{equation}
4-indeces:
\begin{equation}
 M_{iikl}^{\alpha\beta} = 0,
\quad 
 M_{iikl}^{\alpha\beta} =  M_{lkij}^{\beta\alpha} 
\end{equation}

\paragraph{Adimensional tensors:}
\begin{equation}
 \hat{\tens{A}}_{\alpha\beta}
= \frac{\tens{A}_{\alpha\beta}}{3\pi(a_{\alpha} + a_{\beta})},
\quad
 \hat{\tens{B}}_{\alpha\beta}
= \frac{\tens{B}_{\alpha\beta}}{\pi(a_{\alpha}+ a_{\beta})^2},
\quad
 \hat{\tens{C}}_{\alpha\beta}
= \frac{\tens{C}_{\alpha\beta}}{\pi(a_{\alpha}+ a_{\beta})^3}
\end{equation}

\begin{equation}
 \hat{\tens{G}}_{\alpha\beta}
= \frac{\tens{G}_{\alpha\beta}}{3\pi(a_{\alpha} + a_{\beta})},
\quad
 \hat{\tens{H}}_{\alpha\beta}
= \frac{\tens{H}_{\alpha\beta}}{\pi(a_{\alpha}+ a_{\beta})^3},
\quad
 \hat{\tens{M}}_{\alpha\beta}
= \frac{\tens{M}_{\alpha\beta}}{(5/6)\pi(a_{\alpha}+ a_{\beta})^3}
\end{equation}

\paragraph{Variables:}
\begin{equation}
 \bm{r} = \bm{r}_\alpha - \bm{r}_\beta, \quad
 \bm{n} = \frac{\bm{r}}{r}
\end{equation}
\begin{equation}
  s \equiv \frac{2r}{a_{\alpha}+ a_{\beta}}, \quad
 \lambda \equiv \frac{a_\beta}{a_\alpha}, \quad
\xi \equiv s - 2
\end{equation}



\paragraph{Scalar functions}

\begin{equation*}
  \hat{A}_{ij}^{\alpha \beta}
  = 
  X_{\alpha \beta}^{A} n_i n_j
  +
  Y_{\alpha \beta}^{A} (\delta_{ij} - n_i n_j)
 \approx   X_{\alpha \beta}^{A} n_i n_j
\end{equation*}

\begin{equation*}
  \hat{B}_{ij}^{\alpha \beta}
  = 
  Y_{\alpha \beta}^{B}   \varepsilon_{ijk} n_k
\approx 0
\end{equation*}

\begin{equation*}
  \hat{C}_{ij}^{\alpha \beta}
  = 
  X_{\alpha \beta}^{C} n_i n_j
  +
  Y_{\alpha \beta}^{C} (\delta_{ij} - n_i n_j)
 \approx
  X_{\alpha \beta}^{C} n_i n_j
\end{equation*}

\begin{equation*}
 \hat{G}_{ijk}^{\alpha \beta}
 = 
X_{\alpha\beta}^{G} 
\left(n_i n_j - \frac{1}{3}\delta_{ij}\right)n_k
+
Y_{\alpha\beta}^{G}
\left(
n_i \delta_{jk} + n_j \delta_{ik} - 2 n_i n_j n_k
\right)
\approx
X_{\alpha\beta}^{G} 
\left(n_in_j - \frac{1}{3}\delta_{ij}\right)n_k
\end{equation*}
\begin{equation*}
 \hat{H}_{ij}^{\alpha\beta}
 = Y_{\alpha\beta}^{H}
(n_i \varepsilon_{jkm} n_m 
+ n_j \varepsilon_{ikm} n_m )
 \approx 0
\end{equation*}
\begin{align*}
 \hat{M}_{ijkl}^{\alpha\beta}
& = 
\frac{3}{2}
X_{\alpha\beta}^{M}
 \left(e_i e_j
 - \frac{1}{3} \delta_{ij}
\right)
 \left(e_k e_l
 - \frac{1}{3} \delta_{kl}
\right) \\
& \quad
+
\frac{1}{2} 
Y_{\alpha\beta}^{M}
\left(
e_i \delta_{jl} e_k
+
e_j \delta_{il} e_k
+
e_i \delta_{jk} e_l
+
e_j \delta_{ik} e_l
-
4 e_i e_j e_k e_l
\right) \\
&\quad
+ 
\frac{1}{2}
Z_{\alpha\beta}^{M}
\left(
\delta_{ik}\delta_{jl}
+
\delta_{jk}\delta_{il}
-
\delta_{ij}\delta_{kl}
+
e_i e_j \delta_{kl} 
+
\delta_{ij} e_k e_l
+
e_i e_j e_k e_l \right.\\
&\qquad \qquad \quad
\left.
- 
e_i \delta_{jl} e_k
-
e_j \delta_{il} e_k
-
e_i \delta_{jk} e_l
-
e_j \delta_{ik} e_l
\right) \\
&\approx
\frac{3}{2}
 X_{\alpha\beta}^{M}
 \left(e_i e_j
 - \frac{1}{3} \delta_{ij}
\right)
 \left(e_k e_l
 - \frac{1}{3} \delta_{kl}
\right)
\end{align*}

\paragraph{Leading terms}

The leading terms include the factor $\xi^{-1}$.
%
It can be found in $X^{A}$, $X^{G}$ and $X^{M}$.
%



\begin{equation*}
 g_1 (\lambda)
= \frac{2\lambda^2}{(1+\lambda)^3}
\end{equation*}

\begin{equation*}
 X_{\alpha\alpha}^{A}
\approx 
g_1(\lambda) \xi^{-1}
\end{equation*}
\begin{equation*}
 X_{\alpha\beta}^{A}
\approx
-\frac{2}{1+\lambda}g_1(\lambda) \xi^{-1}
 = 
-\frac{2}{1+\lambda} X_{\alpha\alpha}^{A} 
\end{equation*}

\begin{equation*}
 X_{\alpha\alpha}^{C} 
\sim
 X_{\alpha\beta}^{C}
\sim
\mathcal{O}(1) + \mathcal{O}(\xi \ln \xi^{-1}).
\end{equation*}

\begin{equation*}
 X_{\alpha\alpha}^{G}
= \frac{3}{2}g_1(\lambda) \xi^{-1}
= \frac{3}{2} X_{\alpha\alpha}^{A} 
\end{equation*}
\begin{equation*}
 X_{\alpha \beta}^{G}
=
\frac{-6}{(1+\lambda)^2}
g_1 (\lambda) \xi^{-1}
= 
\frac{-6}{(1+\lambda)^2} X_{\alpha\alpha}^{A}
\end{equation*}

\begin{equation*}
 X_{\alpha \alpha}^{M}
=
\frac{3}{5} g_1(\lambda) \xi^{-1}
= \frac{3}{5} X_{\alpha\alpha}^{A}
\end{equation*}
\begin{equation}
 X_{\alpha \beta}^{M}
=
\frac{8\lambda}{(1+\lambda)^3} g_1(\lambda) \xi^{-1}
= 
\frac{8\lambda}{(1+\lambda)^3} X_{\alpha\alpha}^{A}
= 
\frac{4g_1}{\lambda} X_{\alpha\alpha}^{A}
\end{equation}


\newpage

$\bm{E}_{\infty} = 
(E_{xx}^{\infty},
E_{xy}^{\infty},
E_{xz}^{\infty},
E_{yz}^{\infty},
E_{yy}^{\infty}
)
=
(0, 0, \dot{\gamma}/2, 0, 0) $



\begin{align}
 \tilde{\tens{G}}_{\alpha\alpha} \bm{E}_{\infty}
&= 
\begin{pmatrix}
 \tilde{G}^{\alpha\alpha}_{x xx} E_{xx}^{\infty}
+ 2\tilde{G}^{\alpha\alpha}_{x xy} E_{xy}^{\infty}
+ 2\tilde{G}^{\alpha\alpha}_{x xz} E_{xz}^{\infty}
+  \tilde{G}^{\alpha\alpha}_{x yy} E_{yy}^{\infty}
+ 2\tilde{G}^{\alpha\alpha}_{x yz} E_{yz}^{\infty}
+  \tilde{G}^{\alpha\alpha}_{x zz} E_{zz}^{\infty} 
\\
\tilde{G}^{\alpha\alpha}_{y xx} E_{xx}^{\infty}
+ 2\tilde{G}^{\alpha\alpha}_{y xy} E_{xy}^{\infty}
+ 2\tilde{G}^{\alpha\alpha}_{y xz} E_{xz}^{\infty}
+  \tilde{G}^{\alpha\alpha}_{y yy} E_{yy}^{\infty}
+ 2\tilde{G}^{\alpha\alpha}_{y yz} E_{yz}^{\infty}
+  \tilde{G}^{\alpha\alpha}_{y zz} E_{zz}^{\infty} 
\\
 \tilde{G}^{\alpha\alpha}_{z xx} E_{xx}^{\infty}
+ 2\tilde{G}^{\alpha\alpha}_{z xy} E_{xy}^{\infty}
+ 2\tilde{G}^{\alpha\alpha}_{z xz} E_{xz}^{\infty}
+  \tilde{G}^{\alpha\alpha}_{z yy} E_{yy}^{\infty}
+ 2\tilde{G}^{\alpha\alpha}_{z yz} E_{yz}^{\infty}
+  \tilde{G}^{\alpha\alpha}_{z zz} E_{zz}^{\infty} 
\end{pmatrix}
\\
&=
\begin{pmatrix}
 2\tilde{G}^{\alpha\alpha}_{x xz} E_{xz}^{\infty}
\\
 2\tilde{G}^{\alpha\alpha}_{y xz} E_{xz}^{\infty}
\\
 2\tilde{G}^{\alpha\alpha}_{z xz} E_{xz}^{\infty}
\end{pmatrix}
\end{align}

\subsubsection*{Ichiki-san's code}
$s_1 = 2 s / (s^2-4)$, $s_4 = 4 / (s^2-4)$ , $s-2 = \xi $
\begin{align*}
 X^{G}_{\alpha\alpha} &= g_1 s_1 = \frac{2g_1 s }{(s-2)(s+2)} = 
\frac{2g_1 (\xi+2) }{\xi(\xi+4)} \approx  \frac{g_1  }{\xi }  \\
 X^{G}_{\alpha\beta} &= -g_1 s_4 = - \frac{4g_1  }{(s-2)(s+2)}
= - \frac{4g_1  }{\xi(\xi+4)} 
\approx - \frac{ g_1  }{\xi} 
\end{align*}

A crib sheet $\rightarrow$ \texttt{libstokes} written by Ichiki-san
{\footnotesize
\begin{verbatim}
/* calc XG11 and XG12 for lubrication form
 * INPUT
 *  n : max order
 *  l := a2 / a1
 *  s := 2 * r / (a1 + a2)
 * OUTPUT
 *  YC11
 *  YC12
 */
void twobody_XG_lub (int n, double l, double s,
		     double *XG11, double *XG12)
{
  if (n%2 != 0) n--; // make n even for fail-safe

  double *f = (double *)malloc (sizeof (double) * (n + 1));
  CHECK_MALLOC (f, "twobody_XG_lub");

  twobody_XG (n, l, f);

  *XG11 = 0.0;
  *XG12 = 0.0;

  double l1 = 1.0 + l;
  double l13 = l1 * l1 * l1;
  double g1, g2, g3;
  g1 = 3.0 * l * l / l13;
  g2 = 0.3 * l * (1.0 + l * (12.0 - 4.0 * l)) / l13;
  g3 = (5.0 + l*(181.0 + l*(-453.0 + l*(566.0 - 65.0 *l)))) / 140.0 / l13;

  double s1, s2, s3, s4, s5;
  double ls2, ls5;
  s1 = 2.0 * s / (s * s - 4.0);
  s2 = (s + 2.0) / (s - 2.0);
  s3 = (s * s - 4.0) / 4.0;
  s4 = 4.0 / (s * s - 4.0);
  s5 = s * s / (s * s - 4.0);
  ls2 = log (s2);
  ls5 = log (s5);
  
  *XG11 = g1*s1 + (g2 + g3*s3)*ls2 - g3*s;
  // next is m = 1

  *XG12 = -g1*s4 - (g2 + g3*s3)*ls5 + g3;
  // next is m = 2

  double s1l = s * (1.0 + l);
  double s1lm = s1l;

  double s21  = 2.0 / s;
  double s2m  = s21;

  double a;
  int m;
  for (m = 1; m < n; m ++){
      if (m%2 == 0){
        a = - g1 - 2.0 * g2 / (double)m + 4.0 * g3 / (double)(m*(m+2));
        *XG12 -= f[m] / s1lm + a * s2m;
      }else{
        a = - g1 - 2.0 * g2 / (double)m + 4.0 * g3 / (double)(m*(m+2));
        *XG11 += f[m] / s1lm + a * s2m;
      }
      s1lm *= s1l;
      s2m  *= s21;
    }

  *XG12 *= 4.0 / l1 / l1;

  free (f);
}
\end{verbatim}
}

\newpage

\section{Ball\&Melrose model}

One-body limit:
\begin{align}
 \bm{F}_i^{\mathrm{self}} &= 
-6 \pi \mu a ( \bm{v}_i - \bm{U}^{\infty} (\bm{r}_i) ) \\
 \bm{T}_i^{\mathrm{self}} &= 
-8 \pi \mu a^3 ( \bm{\omega}_i - \bm{\Omega}^{\infty})
\end{align}

\citet{Melrose_2004a} uses 
the leading squeeze mode of lubrication force:
\begin{align}
 \bm{F}_{ij}^{\mathrm{pair}}
&=
- \alpha_{\mathrm{n}}(h_{ij})
\bigl\{
(\bm{v}_i - \bm{v}_j)
\cdot \bm{n}_{ij}
\bigr\}
\bm{n}_{ij} \\
 \bm{T}_{ij}^{\mathrm{pair}}
&=
\bm{0},
\end{align}
where
\begin{equation}
 \alpha_{\mathrm{n}}(h_{ij}) = 
\frac{3 \pi \mu a^2}{2 h_{\ij}}.
\end{equation}
Note: We should check the model of \citet{Ball_1997,Melrose_2004a}.
%
We need to understand the difference from FTS version~\citet{Jeffrey_1992}.


Hydrodynamic interaction acting on particle $i$
is given by
\begin{align}
\bm{F}_i^{\mathrm{H}}
&=
\bm{F}_i^{\mathrm{self}}
+
\sum_j
\bm{F}_{ij}^{\mathrm{pair}} \\
\bm{T}_i^{\mathrm{H}}
&=
\bm{T}_i^{\mathrm{self}}
\end{align}


\subsection*{Dimensionless equations}

The unit of length, velocity and force 
are given:
$L_0 = a$,  $v_0 = a \dot{\gamma}$, and $F_0 \equiv 6 \pi \mu a v_0$.

The dimensionless variables are introduced:
$\tilde{h}_{ij} = h_{ij} / L_0$,
$\tilde{\bm{v}}_i = \bm{v}_i / U_0$,
and $\tilde{\bm{F}}_{ij}^{\mathrm{pair}} = \bm{F}_{ij}^{\mathrm{pair}} / F_0$.

The relations are written as follows:
\begin{equation}
 \tilde{\bm{F}}_i^{\mathrm{self}} = 
-( \tilde{\bm{v}}_i - 
\tilde{\bm{U}}^{\infty} 
(\tilde{\bm{r}}_i) )
\end{equation}
\begin{equation}
 \tilde{\bm{F}}_{ij}^{\mathrm{pair}}
= 
- \frac{1}{4 \tilde{h}_{ij}}
\bigl\{
(\tilde{\bm{v}}_i-
\tilde{\bm{v}}_i)\cdot
\bm{n}_{ij}
\bigr\}\bm{n}_{ij}
\end{equation}

\begin{equation}
 \bm{U}^{\infty}(\bm{r})
 = 
 \dot{\gamma} z \bm{e}_{x}
\end{equation}
\begin{equation}
\Longrightarrow
  \tilde{\bm{U}}^{\infty}(\tilde{\bm{r}})
 = 
 \dot{\gamma} z \bm{e}_{x} / U_0
=  \dot{\gamma} z \bm{e}_{x} / (\dot{\gamma} a)
= \tilde{z} \bm{e}_{x} 
\end{equation}


\subsection*{Matrix form}
\begin{align*}
 \bm{F}_{\alpha\beta}^{\mathrm{pair}}
&= 
- \frac{1}{4 h_{\alpha\beta}}
\bigl\{
(\bm{v}_\alpha-
\bm{v}_\beta)\cdot
\bm{n}_{\alpha\beta}
\bigr\}\bm{n}_{\alpha\beta} \\
&=
- \frac{1}{4 h_{\alpha\beta}}
\Bigl[
\bigl\{
(\bm{v}_\alpha- \bm{U}^{\infty}_\alpha)-
(\bm{v}_\beta- \bm{U}^{\infty}_\beta)
+ \bm{U}^{\infty}_\alpha
- \bm{U}^{\infty}_\beta
\bigr\}\cdot
\bm{n}_{\alpha\beta}
\Bigr]
\bm{n}_{\alpha\beta}
\end{align*}


$\Delta \bm{v} \equiv (\bm{v}- \bm{U}^{\infty})$
\begin{align*}
\Delta \bm{v} \cdot
\bm{n} \bm{n}
= &
(\Delta v_x n_x 
+\Delta v_y n_y
+\Delta v_z n_z) n_x \bm{i}\\
& 
+ (\Delta v_x n_x 
+\Delta v_y n_y
+\Delta v_z n_z) n_y \bm{j}\\
&
+ (\Delta v_x n_x 
+\Delta v_y n_y
+\Delta v_z n_z) n_z \bm{k} \\
=&
\begin{pmatrix}
 n_x n_x &   n_y n_x &   n_z n_x \\
 n_x n_y &   n_y n_y &   n_z n_y \\
 n_x n_z &   n_y n_z &   n_z n_z 
\end{pmatrix}
\begin{pmatrix}
 \Delta v_x \\
 \Delta v_y \\
 \Delta v_z 
\end{pmatrix} 
\end{align*}
\begin{equation*}
(\Delta \bm{v} \cdot
\bm{n} \bm{n})_i
= n_i n_j \Delta v_j
\end{equation*}


\begin{align*}
  \bm{F}_\alpha  &= 
  -  \Delta \bm{v}_\alpha
  - \sum_{\beta}
  \frac{1}{4h_{\alpha\beta}}
  \left(
    \bm{n}_{\alpha\beta} (\bm{n}_{\alpha\beta}\cdot \Delta \bm{v}_{\alpha})
    - 
    \bm{n}_{\alpha\beta} (\bm{n}_{\alpha\beta}\cdot \Delta \bm{v}_\beta )
    +
    \bm{n}_{\alpha\beta}
    \left\{
      \bm{n}_{\alpha\beta}  \cdot
      (\bm{U}_{\alpha}^{\infty}-\bm{U}_{\beta}^{\infty})
    \right\}
  \right)  \\
&=  -  \Delta \bm{v}_\alpha 
 - \sum_{\beta}
  \frac{1}{4h_{\alpha\beta}}
  \left(
    \bm{n}_{\alpha\beta} (\bm{n}_{\alpha\beta}\cdot \Delta \bm{v}_{\alpha})
    - 
    \bm{n}_{\alpha\beta} (\bm{n}_{\alpha\beta}\cdot \Delta \bm{v}_\beta )
    +
    \bm{n}_{\alpha\beta}
    \left\{
  ....
    \right\}
  \right) \\
&=  -  \Delta \bm{v}_\alpha 
 - \sum_{\beta}
  \frac{1}{4h_{\alpha\beta}}
  \left(
    \bm{n}_{\alpha\beta} (\bm{n}_{\alpha\beta}\cdot \Delta \bm{v}_{\alpha})
    - 
    \bm{n}_{\alpha\beta} (\bm{n}_{\alpha\beta}\cdot \Delta \bm{v}_\beta )
    +
r_{\alpha\beta}
  \bm{n}_{\alpha\beta}
  \left\{
\bm{n}_{\alpha\beta} \cdot
(\tens{E}^{\infty} \bm{n}_{\alpha\beta})
    \right\} 
  \right) 
\end{align*}

\begin{align*}
    \bm{n}_{\alpha\beta}    \left\{
  ....
    \right\} &= 
  \bm{n}_{\alpha\beta}
  \left\{
    \bm{n}_{\alpha\beta}  \cdot
      (
\Omega^{\infty} \times (\bm{r}^{\alpha}-\bm{r}^{\beta})
+ 
\tens{E}^{\infty}
(\bm{r}^{\alpha}-\bm{r}^{\beta}))
)
    \right\} \\
&= 
r_{\alpha\beta}
  \bm{n}_{\alpha\beta}
  \left\{
      (
\bm{n}_{\alpha\beta} \cdot \Omega^{\infty} \times \bm{n}_{\alpha\beta}
+ 
\bm{n}_{\alpha\beta} \cdot
(\tens{E}^{\infty} \bm{n}_{\alpha\beta})
)
    \right\} \\
&= 
r_{\alpha\beta}
  \bm{n}_{\alpha\beta}
  \left\{
\bm{n}_{\alpha\beta} \cdot
(\tens{E}^{\infty} \bm{n}_{\alpha\beta})
    \right\} 
\end{align*}

\begin{align*}
 F^{\alpha}_i
&=
- \Delta v^{\alpha}_i
- \frac{1}{4 h}
\left(
n_{i} n_j \Delta v^{\alpha}_j
- n_{i} n_j \Delta v^{\beta}_j
+ r n_i n_j n_k E_{jk}
\right) \\
&=
- \Delta v^{\alpha}_i
- \frac{1}{4 h}
\left(
n_{i} n_j \Delta v^{\alpha}_j
- n_{i} n_j \Delta v^{\beta}_j
+ r n_i n_j n_k E_{jk}
\right) \\
&=
- \Delta v^{\alpha}_i
- \frac{1}{4 h}
\left(
n_{i} n_j \Delta v^{\alpha}_j
- n_{i} n_j \Delta v^{\beta}_j
+  r n_i n_x n_z 
\right)
\end{align*}
In the last line,
$E_{xz}=E_{zx}=1/2$ is used.

\begin{equation}
 \bm{F}^{\alpha} = - \Delta \bm{v}^{\alpha}
- \frac{1}{4h} (\bm{n}\bm{n}\cdot\Delta \bm{v}^{\alpha}
- \bm{n}\bm{n}\cdot\Delta \bm{v}^{\beta}
 + r  n_x n_z \bm{n}
)
\end{equation}





\begin{align}
\begin{pmatrix}
\vdots \\ \bm{F}_i \\ \vdots \\ \bm{F}_j \\  \vdots  
\end{pmatrix}
& =
\begin{pmatrix}
\vdots \\
\bm{F}_{i}^{\mathrm{self}} +  \bm{F}_{ij}^{\mathrm{pair}}  
\\ \vdots \\ 
\bm{F}_{j}^{\mathrm{self}} +  \bm{F}_{ji}^{\mathrm{pair}}  
\\  \vdots  
\end{pmatrix} \\
&=
-
\begin{pmatrix}
\cdots & \cdots & \cdots & \cdots & \cdots\\
\cdots & 1 + \frac{1}{4h_{ij}}\bm{n}_{ij} \otimes \bm{n}_{ij}  
& \cdots & -\frac{1}{4h_{ij}}\bm{n}_{ij} \otimes \bm{n}_{ij}   & \cdots\\
\cdots & \cdots & \cdots & \cdots & \cdots\\
\cdots & -\frac{1}{4h_{ji}}\bm{n}_{ji} \otimes \bm{n}_{ji} & \cdots 
& 1 + \frac{1}{4h_{ji}}\bm{n}_{ji} \otimes \bm{n}_{ji} 
 & \cdots \\
\cdots & \cdots & \cdots & \cdots & \cdots 
\end{pmatrix}
\begin{pmatrix}
\vdots \\
\Delta \bm{v}_i \\ 
\vdots \\
\Delta \bm{v}_j \\
\vdots 
\end{pmatrix} \\
& \quad 
+ 
\begin{pmatrix}
\vdots \\
-\frac{1}{4h_{ij}}
 \frac{(z_i - z_j )(x_i-x_j)}{r_{ij}} 
 \bm{n}_{ij} \\
\vdots \\
-\frac{1}{4h_{ji}} 
 \frac{(z_j - z_i )(x_j - x_i)}{r_{ji}} 
\bm{n}_{ji} \\
\vdots 
\end{pmatrix} 
\end{align}

\bibliography{/Users/seto/Dropbox/Papers/rse}
\end{document}