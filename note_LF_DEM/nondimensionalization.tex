\RequirePackage[l2tabu, orthodox]{nag}
\documentclass[fontsize=11pt]{scrartcl}
\usepackage{amsmath,amsfonts,amsthm}
\usepackage{bm}
\usepackage[a4paper]{geometry}
\usepackage[protrusion=true,expansion=true]{microtype}
\usepackage{siunitx}
\usepackage{xcolor}
\usepackage[pdftex]{graphicx}
\usepackage[pdftex,bookmarks,colorlinks]{hyperref}
\usepackage[english]{babel}
\usepackage[T1]{fontenc}
\usepackage[utf8]{inputenc}
\usepackage{newtxtext}
\usepackage[varg]{newtxmath}
\usepackage{booktabs}
\usepackage{datetime}
\usepackage{setspace}
\usepackage{enumitem}
\usepackage{cleveref}
\usepackage[numbers]{natbib}
\bibliographystyle{abbrvnat}
\setlist{itemsep=-3pt}
\title{Nondimentionalization}
\date{\shortdate\today \, \ampmtime }
\author{seto}
\begin{document}
\maketitle

\subsection*{Force balance equation}

The following force balance equation is considered here;
\begin{equation}
 \bm{F}_{\mathrm{H}} +  \bm{F}_{\mathrm{R}}  = 0.
\end{equation}
%
The hydrodynamic interaction is
assumed to be linear form of velocities.
%
\begin{equation}
 \bm{F}_{\mathrm{H}} = - \bm{R}\cdot\bm{U}.
\end{equation}

\subsection*{Representative quantities}

\subsubsection*{From shear rate}
Particle radius is commonly considered
as the representative length in
our particle simulations: $L^{\ast} = a$.

One way to give the representative particle velocity
in a sheared suspension is
\begin{equation}
U_{\mathrm{H}}^{\ast} = a \dot{\gamma}.
\end{equation}
%
%
The typical hydrodynamic force is given
by the typical velocity $U_{\mathrm{H}}^{\ast} $
and Stokes-drag form;
\begin{equation}
  F_{\mathrm{H}}^{\ast} = 6 \pi \eta_0 a U_{\mathrm{H}}^{\ast}.
\end{equation}
Here, we indicate nondimensionalized variables
in the hydrodynamic unit by using `tilde',
i.e.,
$\bm{F}_{\mathrm{H}} = F_{\mathrm{H}}^{\ast} \tilde{\bm{F}}_{\mathrm{H}} $,
etc.

\subsubsection*{From repulsive force}

The repulsive force $\bm{F}_{\mathrm{R}}$ 
is a function of particle-to-particle distance.
%
We take the amplitude $F_{\mathrm{R}}^{\ast}$
to represent the typical strength:
%
\begin{equation}
 \bm{F}_{\mathrm{R}} =  F_{\mathrm{R}}^{\ast}
\hat{\bm{F}}_{\mathrm{R}},
\end{equation}
where `hat' indicates nondimensionalized variables
based on this repulsive force.
%
The corresponding velocity can also be
given via the Stokes-drag form;
\begin{equation}
 U^{\ast}_{\mathrm{R}}
  = \frac{F_{\mathrm{R}}^{\ast}}{6\pi\eta_0 a}.
\end{equation}
%

\subsection*{Hydrodynamic (or shear-rate) unit}

In the hydrodynamic unit, the nondimensionalized
force balance equation is
\begin{equation}
 \frac{\bm{F}_{\mathrm{H}}}{F^{\ast}_{\mathrm{H}}}
  +
  \frac{\bm{F}_{\mathrm{R}}}{F^{\ast}_{\mathrm{H}}}  = 0
  \quad
  \Longrightarrow 
    \quad
 - \bm{R}' \cdot \tilde{\bm{U}} + \tilde{\bm{F}}_{\mathrm{R}} = 0
\end{equation}
where
$\bm{R}' \equiv \bm{R} / 6 \pi \eta_0 a $.

In order to see the shear rate dependence,
it is convenient to represent
the typical repulsive force $F_{\mathrm{R}}^{\ast}$.
%
\begin{equation}
 - \tilde{\bm{R}} \cdot \tilde{\bm{U}} +
\frac{1}{\alpha}  \hat{\bm{F}}_{\mathrm{R}} = 0
\end{equation}
%
This $\alpha$ is the nondenominational shear rate:
\begin{equation}
 \alpha \equiv
\frac{\bm{F}_{\mathrm{H}}^{\ast}}{\bm{F}_{\mathrm{R}}^{\ast}}
=
  \frac{\dot{\gamma}}{F_{\mathrm{R}}^{\ast}/6\pi\eta_0 a^2}\label{102734_31May15}
\end{equation}

The unit of time is
\begin{equation}
 t_{\mathrm{H}}^{\ast} \equiv \frac{L^{\ast}}{U_{\mathrm{H}}^{\ast}}
  = \frac{1}{\dot{\gamma}}.
\end{equation}
Therefore, the nondenominational time $\tilde{t}$
is shear strain (when shear rate is constant):
\begin{equation} 
 \tilde{t} = \frac{t}{t_{\mathrm{H}}^{\ast}} = t \dot{\gamma}
\end{equation}

[note]
We call this unit system ``hydrodynamic''.
But, the essential origin for this unit system
is the way to give typical velocity,
i.e., $U^{\ast} = a \dot{\gamma}$.
%
``Simple shear'' unit would be more proper name.


\subsection*{Repulsive force unit}

In the repulsive unit,
the nondimensionalized
force balance equation is
\begin{equation}
 \frac{\bm{F}_{\mathrm{H}}}{F^{\ast}_{\mathrm{R}}}
  +
  \frac{\bm{F}_{\mathrm{R}}}{F^{\ast}_{\mathrm{R}}}  = 0
  \Longrightarrow
    \quad
    - \bm{R}' \cdot \hat{\bm{U}} +
    \hat{\bm{F}}_{\mathrm{R}} = 0
\end{equation}
The time unit is
\begin{equation}
 t_{\mathrm{R}}^{\ast}
  \equiv \frac{L^{\ast}}{U_{\mathrm{R}}^{\ast}}
  = \frac{6 \pi \eta_0 a^2}{F_{\mathrm{R}}^{\ast}}
\end{equation}

In rate controlled simulations,
the relation between
the nondimensionalized time
$\hat{t} \equiv t /t_{\mathrm{R}}^{\ast}$
and shear strain is
\begin{equation}
 \gamma
  \equiv \dot{\gamma} t 
  = \dot{\gamma} t_{\mathrm{R}}^{\ast} \hat{t} 
   = 
\frac{F_{\mathrm{H}}^{\ast}}{F_{\mathrm{R}}^{\ast}}
\hat{t} = \alpha \hat{t}
\end{equation}
where $\alpha$ is defined in \eqref{102734_31May15}.


\end{document}