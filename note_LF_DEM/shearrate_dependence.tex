%\documentclass[
%  aip,author-year,
%  preprint,
%  %unsortedaddress,
%  amsmath,amssymb,
%  %linenumbers,
%  tightenlines
%]{revtex4-1}
\documentclass{article}
\usepackage{amsmath,amssymb}
\usepackage{bm}
%\usepackage{hyperref}
\usepackage[pdftex]{graphicx}	% required for `\includegraphics' (yatex added)
\usepackage{datetime}
\usepackage{natbib}
\bibliographystyle{unsrtnat}
\newcommand{\figref}[1]{\figurename ~\ref{#1}}
\begin{document}
\title{Shear rate dependence of no-Brownian suspension}
\date{\shortdate\today \, \ampmtime }
\maketitle
\subsection*{Units}

Our first target
is rheology of non-Brownian suspensions
under simple shear flows
(shear rates, $\dot{\gamma}$).
%
It is convenient
to use units of length, velocity and force
as follows:
\begin{gather}
 L_0 = a_0,  \\
 U_0 = L_0 \dot{\gamma}, \\
 F_0 = 6 \pi \eta_0 a U_0 = 6 \pi \eta_0 a^2 \dot{\gamma}.
\end{gather}
$a_0$ is the radius of particles
(smaller one, if it is bidisperse, i.e., $a_0 < a_1$).

%%%%%%%%%%%%%%%%%%%%%%%%%%%%%%%%%%%%%%%%%%%%%%%%%%%%%%%%%%%%%%%%%%%%%%%%%%%%%%%%%%%%%%%%%%%%%%%%%%%%

Non-dimensional variables are denoted 
by hat (~$\hat{}$~):
\begin{align}
 \hat{F} &=  F/F_0, \\
 \hat{T} &=  T/(F_0 L_0), \\
 \hat{S} &=  S/(F_0 L_0),
\end{align}
and $ \hat{U} = U / U_0$,
$\hat{\Omega} = \Omega $,
and $\hat{E} = E L_0 / U_0 $.
%

%%%%%%%%%%%%%%%%%%%%%%%%%%%%%%%%%%%%%%%%%%%%%%%%%%%%%%%%%%%%%%%%%%%%%%%%%%%%%%%%%%%%%%%%%%%%%%%%%%%%

Due to the linearity of Stokes regime,
hydrodynamic interaction
can be written as non-dimensional model,
where the shear rate 
does not appear in the model.
%
We can scale all non-dimensional results
by the shear rate.


\subsection*{The finite lubrication}

Our working hypothesis is
\emph{to use finite lubrication}.
%
For monodisperse system,

%
\begin{equation}
  \bm{F}_{\mathrm{lub}}^{(i,j)}
  = -\alpha (h)
  (\bm{v}^{(i)} - \bm{v}^{(j)})\cdot
  \bm{n}^{(i,j)} \bm{n}^{(i,j)}.\label{123922_1Mar13}
\end{equation}
\begin{equation}
  \alpha(h)
  = 
  \begin{cases}
    3 \pi \eta_0 a^2 / 2(h+c) & h > 0 \\
    3 \pi \eta_0 a^2 / 2c & h\leq 0 
  \end{cases}\label{124138_1Mar13}
\end{equation}
%
This way to introduce the finiteness
is not obvious.
%
We might imagine smaller spheres
with a radius $b < a$.
%
We may consider them as hydrodynamic radii.
%
Each particle have 
both geometrical and hydrodynamic radii: $a$ and $b$.
%
When the gap is geometrically $h \equiv r - 2a$,
the hydrodynamic gap is
\begin{equation}
  h' = 
  r - 2b =
  r - 2a + 2(a-b) = h + c.
\end{equation}
where $c \equiv 2(a-b)$.
%
So, the eqs.~\eqref{123922_1Mar13} 
and \eqref{124138_1Mar13} for $h>0$
is considered as
a solution of Stokes equations
for two spheres of radius $b$.

%%%%%%%%%%%%%%%%%%%%%%%%%%%%%%%%%%%%%%%%%%%%%%%%%%%%%%%%%%%%%%%%%%%%%%%%%%%%%%%%%%%%%%%%%%%%%%%%%%%%
The non-dimensional form
is 
\begin{equation}
  \hat{\bm{F}}_{\mathrm{lub}}^{(i,j)}
  = \hat{\alpha} (\hat{h})
  (\hat{\bm{v}}^{(i)} - \hat{\bm{v}}^{(j)})\cdot
  \bm{n}^{(i,j)} \bm{n}^{(i,j)}.
\end{equation}
\begin{equation}
  \hat{\alpha}(\hat{h})
  = 
  \begin{cases}
    1 / 4(\hat{h}+\hat{c}) & \hat{h} > 0 \\
    1 /4\hat{c} & \hat{h} \leq 0 
  \end{cases}
\end{equation}
%
If the above interpretation is ok 
to have the finiteness in 
the lubrication model,
the shear-rate independence
of the Stokes regime remains
in the lubrication model.
%
The trajectories of particles before the first contact
never depend on shear rates.
%
Just this finiteness
allows particles 
to come into contact.

\subsection*{Contact force model}

Contact forces act between
two particles $i$ and $j$,
when the distance $r$
is smaller than $r_0 = a_i + a_j$.

\paragraph{Rigid sphere model}

Typical hydrodynamic forces
are much smaller than
the force which can deform particles.
%
Young's modulus of particles
(1--100~GPa) is usually large enough
to consider the particles are rigid body
in sheared suspensions.
%
In the rigid-body limit of elastic body,
any force cannot cause finite deformation.
%
In simulations (such as DEM),
we need to use a finite potential function
to mimic this behavior.
%
As mentioned above,
we can model the hydrodynamic interaction part
as shear-rate independent.
%
To combine models of hydrodynamic and
contact interactions,
there are two possibilities.
%
The one is shear-rate independent contact force model,
and the second is shear-rate scaling contact force model.
%

Since contact force model
is unrelated to shear flows,
the fist one seems reasonable.
%
If so,
in the non-denationalized simulation,
the contact force looks to be scaled by shear rates:
\begin{equation}
  \hat{F}_{\mathrm{c}}(\hat{r}) = \frac{F_{\mathrm{c}}(r) }{F_0}
  = \frac{F_{\mathrm{c}}(\hat{r}L_0) }{6 \pi \eta_0 L_0^2  } 
  \frac{1}{\dot{\gamma}},
\end{equation}
%
where we use the same $F_{\mathrm{c}}(r)$.
Thus,
the hydrodynamic parts 
are equivalent for different shear rates,
while
the shear rate changes the contact force model
in the non-denationalized simulation.
%
We need to check
this model can mimic the rigid spheres
for a wide range of shear rates.
%


%%%%%%%%%%%%%%%%%%%%%%%%%%%%%%%%%%%%%%%%%%%%%%%%%%%%%%%%%%%%%%%%%%%%%%%%%%%%%%%%%%%%%%%%%%%%%%%%%%%%

In the rigid-body limit,
we may expect
no shear-rate dependence
in the non-Brownian+finite-lubrication suspension rheology.
%
If so, 
we should use
a shear-rate independent dimensionless simulation model
by introducing shear-rate scaling contact model.
%
We can just use the same contact model
$\hat{F}_c(\hat{r})$
in the non-denationalized simulation.
%
It means that 
we consider 
different contact models
for different shear rates;
\begin{equation}
 F_{\mathrm{c}}(r) = \hat{F}_{\mathrm{c}}(\hat{r}) F_0
= 6 \pi \eta L_0^2 \dot{\gamma} 
\hat{F}_{\mathrm{c}}(r/L_0).
\end{equation}
%
Since the absence of the shear rate in the code,
it guarantees to have shear-rate independent results,
which is expected in the rigid-bory limit.
%
If friction is the origin
to yield shear-rate dependence,
it is better to use
this shear-rate independent form.
%

\paragraph{Friction model}

Friction laws can be matter.
%
The well-known formula
\begin{equation}
 F_{\|} = \mu F_{\perp}
\end{equation}
is a linear relation.
%
In the non-dimensional form,
shear-rate dependence does not appear:
\begin{equation}
 \hat{F}_{\|} F_0 = \mu \hat{F}_{\perp} F_0
\quad
\rightarrow 
\quad
 \hat{F}_{\|} = \mu \hat{F}_{\perp}.
\end{equation}
%


Cohesion term is also expected:
\begin{equation}
 F_{\parallel} = \mu F_{\perp}
 + \sigma A
\end{equation}
The textbook of Israelachvili says
\begin{quote}
At low loads
the friction force is dominated
by the adhesion contribution,
but at high loads,
where $A \propto F_{\perp}^{2/3}$ ,
it is dominated by the load-dependent contribution 
$\mu F_{\perp}$.
\end{quote}
%
\begin{align}
& 
\hat{F}_{\parallel} F_0
= \mu \hat{F}_{\perp} F_0 +
\sigma (\hat{F}_{\perp} F_0)^{2/3} \\
&
\to \quad
\hat{F}_{\parallel} 
= \mu \hat{F}_{\perp}  +
(\sigma F_0^{-1/3}) \hat{F}_{\perp}^{2/3}  
\end{align}
%
This means that
the cohesion term is less and less important
at higher shear rate.






\bibliography{/Users/seto/Dropbox/Papers/rse}
\end{document}










